\documentclass[11pt]{article}

\usepackage[letterpaper, margin=0.9in]{geometry}
\usepackage{hyperref}
\usepackage{amsmath, amsfonts, amssymb}

\begin{document}
    \title{CS 242 Week 1 Rubrics}
    \author{Zhengyao Lin}
    \date{November, 2019}
    \maketitle
    
    \section{Abstract}
        In week 1, \texttt{yterm} is going to have basic rendering functionality and will support correct interaction and display of most basic command line programs.
    
    \section{Rubrics}
        \begin{itemize}
            \item (10\%) Correct display of the terminal screen with configurable size (columns and rows).
            \item (10\%) Correct display of the (blinking) cursor.
            \item (10\%) Correct parsing of ANSI escape sequences (ignoring color/graphics related sequences now) and separate printable text with control sequences.
            \item (20\%) Support for basic cursor moving sequences (up/down/right/left/carriage return/backspace).
            \item (10\%) Support for character inserting/deleting.
            \item (10\%) Support for scrolling.
            \item (20\%) Optimize performance and compatibility with \texttt{bash} and other commonly used utilities (\texttt{ls}, \texttt{python}, etc.).
            \item (10\%) Basic test infrastructure for ANSI escape sequences and terminal (will add more complete tests for ANSI sequence next week since I'm still pondering how to implement ANSI sequence in a more organized way).
        \end{itemize}
    
    \section{Extra credits :-)}
        \begin{itemize}
            \item (5\%) GitLab CI/CD setup.
        \end{itemize}
\end{document}
